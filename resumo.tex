\section*{Resumo}

Este artigo investiga o papel do Campo Informacional na sustentação das Comunidades Sistêmicas, integrando referenciais teóricos interdisciplinares com vivências práticas do Instituto Lichtara. A pesquisa parte da premissa de que as comunidades humanas são organizadas não apenas por estruturas formais, mas por um tecido invisível de informação e consciência que conecta pessoas e orienta ações coletivas. Através de uma metodologia qualitativa e fenomenológica, que combina revisão bibliográfica, observação participante e análise de práticas vibracionais, o estudo articula teorias de David Bohm (ordem implicada), Rupert Sheldrake (ressonância mórfica), Ervin Laszlo (campo akáshico), Teilhard de Chardin (noosfera) e Niklas Luhmann (sistemas sociais) com experiências diretas de organização comunitária. Os resultados esperados incluem a identificação de padrões de ressonância coletiva, a proposição de um modelo conceitual para Comunidades Sistêmicas e a validação metodológica de práticas vibracionais como instrumentos de pesquisa. O estudo conclui que o consentimento consciente — o SIM — constitui a chave integradora que permite ao Campo Informacional manifestar-se no indivíduo e organizar o coletivo, posicionando o Instituto Lichtara como laboratório vivo de experimentação científico-espiritual.

\textbf{Palavras-chave:} Campo Informacional, Comunidades Sistêmicas, Consciência Coletiva, Ressonância Mórfica, Instituto Lichtara.

\newpage
